\documentclass[12pt, a4paper, utf8]{article}
\usepackage[utf8]{inputenc}
\usepackage[english, russian]{babel}
\usepackage[T2A]{fontenc}
%\usepackage{xltxtra}
%\usepackage{xecyr}

%\usepackage{fontspec}
%\defaultfontfeatures{Scale=MatchLowercase}
%\setmainfont[Mapping=tex-text]{Liberation Serif}
%\setsansfont[Mapping=tex-text]{Liberation Sans}
%\setmonofont{Droid Sans Mono}

\usepackage{multicol}

% меняем размеры страницы
\usepackage{geometry}
\geometry{left=3.0cm}
\geometry{right=2.0cm}
\geometry{top=2.0cm}
\geometry{bottom=2.0cm}

\makeindex

% будем считать что переполнения не было, если строка вышла за границу не более чем 0.01pt
\hfuzz=0.001pt

% разрешаю увеличить расстояние между словами не больше чем на 14pt, что бы не было переполнения
\emergencystretch=14pt

% изменение межстрочного интервала
\def\heightline{1.3}
\linespread{\heightline} % 1.3 - это полуторный

%%%%%%%%%%%%%%%%%%%%%%%%%%%%%%%%%%%%%%%%%%%%%%%%
%%% добавляем точки в Оглавление

\renewcommand{\thesection}{\arabic{section}.} 
\renewcommand{\thesubsection}{\arabic{section}.\arabic{subsection}.} 
\renewcommand{\thesubsubsection}{\arabic{section}.\arabic{subsection}.\arabic{subsubsection}.} 

\begin{document}

\begin{titlepage}
\begin{center}
{Министерство образования и науки Российской Федерации\\ 
Государственное образовательное учреждение \\ 
высшего профессионального образования \\ 
<<Нижегородский государственный университет им. Н.И. Лобачевского>>\\

\bf{Факультет вычислительной математики и кибернетики\\
Кафедра математического обеспечения ЭВМ\\
}
\vspace{5em}
АСПИРАНТСКАЯ ДИССЕРТАЦИЯ\\
}
{\bf \Large	\textsf{ Оптимизация значений параметров в системах компьютерного зрения \\ } }
\end{center}
\vspace{3em}
\begin{multicols}{2}
\vbox to 14 cm{\ }
\noindent 
{\bf Допущена к защите} : {\hrulefill}\\
Заведующий кафедрой МО ЭВМ \\
д.ф.--м.н., проф. \\

\vspace*{1em}
{ \noindent
\hspace*{1.5cm}\hbox to 0cm{\raisebox{-1em}{\small Подпись}}\hspace{-1.5cm}{\hrulefill\ \ Стронгин Р. Г. }
}

\vspace*{1em}
{ \noindent
\hbox to 0cm{\raisebox{0.25em}{<<\qquad \qquad \quad >>}}{\hrulefill\ 2011 г. }
}

\noindent
{\bf Выполнил}: аспирант  \\

\vspace*{0.5em}
{ \noindent
\hspace*{1.5cm}\hbox to 0cm{\raisebox{-1em}{\small Подпись}}\hspace{-1.5cm}{\hrulefill\ \ Морозов А. С. }
}

\vspace{1.5em}
\noindent
{\bf Научный руководитель}: \\
д. т. н., профессор кафедры МО ЭВМ \\

{ \noindent
\hspace*{1.5cm}\hbox to 0cm{\raisebox{-1em}{\small Подпись}}\hspace{-1.5cm}{\hrulefill\ \ ФИО }
}
\end{multicols}
\vfill 
\begin{center} 
{\rm Нижний Новгород \\ 2011 г.} 
\end {center} 
\end{titlepage} 

\tableofcontents

\newpage

\section*{Введение}
\addcontentsline{toc}{section}{Введение} 

Введение.

\section{top}

\section*{Заключение}
\addcontentsline{toc}{section}{Заключение} 

Заключение.


\newpage
\begin{thebibliography}{99}
\addcontentsline{toc}{section}{Список литературы}

	\bibitem{rdtscp} Gabriele Paoloni. How to Benchmark Code Execution Times on Intel IA-32 and IA-64 Instruction Set Architectures, September 2010


\end{thebibliography}

\newpage
\addcontentsline{toc}{section}{Список иллюстраций}
\listoffigures

\newpage
\addcontentsline{toc}{section}{Список таблиц}
\listoftables

%\newpage
%\addcontentsline{toc}{section}{Предметный указатель}
%\documentclass[12pt, a4paper, utf8]{article}
\usepackage[utf8]{inputenc}
\usepackage[english, russian]{babel}
\usepackage[T2A]{fontenc}
%\usepackage{xltxtra}
%\usepackage{xecyr}

%\usepackage{fontspec}
%\defaultfontfeatures{Scale=MatchLowercase}
%\setmainfont[Mapping=tex-text]{Liberation Serif}
%\setsansfont[Mapping=tex-text]{Liberation Sans}
%\setmonofont{Droid Sans Mono}

\usepackage{multicol}

% меняем размеры страницы
\usepackage{geometry}
\geometry{left=3.0cm}
\geometry{right=2.0cm}
\geometry{top=2.0cm}
\geometry{bottom=2.0cm}

\makeindex

% будем считать что переполнения не было, если строка вышла за границу не более чем 0.01pt
\hfuzz=0.001pt

% разрешаю увеличить расстояние между словами не больше чем на 14pt, что бы не было переполнения
\emergencystretch=14pt

% изменение межстрочного интервала
\def\heightline{1.3}
\linespread{\heightline} % 1.3 - это полуторный

%%%%%%%%%%%%%%%%%%%%%%%%%%%%%%%%%%%%%%%%%%%%%%%%
%%% добавляем точки в Оглавление

\renewcommand{\thesection}{\arabic{section}.} 
\renewcommand{\thesubsection}{\arabic{section}.\arabic{subsection}.} 
\renewcommand{\thesubsubsection}{\arabic{section}.\arabic{subsection}.\arabic{subsubsection}.} 

\begin{document}

\begin{titlepage}
\begin{center}
{Министерство образования и науки Российской Федерации\\ 
Государственное образовательное учреждение \\ 
высшего профессионального образования \\ 
<<Нижегородский государственный университет им. Н.И. Лобачевского>>\\

\bf{Факультет вычислительной математики и кибернетики\\
Кафедра математического обеспечения ЭВМ\\
}
\vspace{5em}
АСПИРАНТСКАЯ ДИССЕРТАЦИЯ\\
}
{\bf \Large	\textsf{ Оптимизация значений параметров в системах компьютерного зрения \\ } }
\end{center}
\vspace{3em}
\begin{multicols}{2}
\vbox to 14 cm{\ }
\noindent 
{\bf Допущена к защите} : {\hrulefill}\\
Заведующий кафедрой МО ЭВМ \\
д.ф.--м.н., проф. \\

\vspace*{1em}
{ \noindent
\hspace*{1.5cm}\hbox to 0cm{\raisebox{-1em}{\small Подпись}}\hspace{-1.5cm}{\hrulefill\ \ Стронгин Р. Г. }
}

\vspace*{1em}
{ \noindent
\hbox to 0cm{\raisebox{0.25em}{<<\qquad \qquad \quad >>}}{\hrulefill\ 2011 г. }
}

\noindent
{\bf Выполнил}: аспирант  \\

\vspace*{0.5em}
{ \noindent
\hspace*{1.5cm}\hbox to 0cm{\raisebox{-1em}{\small Подпись}}\hspace{-1.5cm}{\hrulefill\ \ Морозов А. С. }
}

\vspace{1.5em}
\noindent
{\bf Научный руководитель}: \\
д. т. н., профессор кафедры МО ЭВМ \\

{ \noindent
\hspace*{1.5cm}\hbox to 0cm{\raisebox{-1em}{\small Подпись}}\hspace{-1.5cm}{\hrulefill\ \ ФИО }
}
\end{multicols}
\vfill 
\begin{center} 
{\rm Нижний Новгород \\ 2011 г.} 
\end {center} 
\end{titlepage} 

\tableofcontents

\newpage

\section*{Введение}
\addcontentsline{toc}{section}{Введение} 

Введение.

\section{top}

\section*{Заключение}
\addcontentsline{toc}{section}{Заключение} 

Заключение.


\newpage
\begin{thebibliography}{99}
\addcontentsline{toc}{section}{Список литературы}

	\bibitem{rdtscp} Gabriele Paoloni. How to Benchmark Code Execution Times on Intel IA-32 and IA-64 Instruction Set Architectures, September 2010


\end{thebibliography}

\newpage
\addcontentsline{toc}{section}{Список иллюстраций}
\listoffigures

\newpage
\addcontentsline{toc}{section}{Список таблиц}
\listoftables

%\newpage
%\addcontentsline{toc}{section}{Предметный указатель}
%\documentclass[12pt, a4paper, utf8]{article}
\usepackage[utf8]{inputenc}
\usepackage[english, russian]{babel}
\usepackage[T2A]{fontenc}
%\usepackage{xltxtra}
%\usepackage{xecyr}

%\usepackage{fontspec}
%\defaultfontfeatures{Scale=MatchLowercase}
%\setmainfont[Mapping=tex-text]{Liberation Serif}
%\setsansfont[Mapping=tex-text]{Liberation Sans}
%\setmonofont{Droid Sans Mono}

\usepackage{multicol}

% меняем размеры страницы
\usepackage{geometry}
\geometry{left=3.0cm}
\geometry{right=2.0cm}
\geometry{top=2.0cm}
\geometry{bottom=2.0cm}

\makeindex

% будем считать что переполнения не было, если строка вышла за границу не более чем 0.01pt
\hfuzz=0.001pt

% разрешаю увеличить расстояние между словами не больше чем на 14pt, что бы не было переполнения
\emergencystretch=14pt

% изменение межстрочного интервала
\def\heightline{1.3}
\linespread{\heightline} % 1.3 - это полуторный

%%%%%%%%%%%%%%%%%%%%%%%%%%%%%%%%%%%%%%%%%%%%%%%%
%%% добавляем точки в Оглавление

\renewcommand{\thesection}{\arabic{section}.} 
\renewcommand{\thesubsection}{\arabic{section}.\arabic{subsection}.} 
\renewcommand{\thesubsubsection}{\arabic{section}.\arabic{subsection}.\arabic{subsubsection}.} 

\begin{document}

\begin{titlepage}
\begin{center}
{Министерство образования и науки Российской Федерации\\ 
Государственное образовательное учреждение \\ 
высшего профессионального образования \\ 
<<Нижегородский государственный университет им. Н.И. Лобачевского>>\\

\bf{Факультет вычислительной математики и кибернетики\\
Кафедра математического обеспечения ЭВМ\\
}
\vspace{5em}
АСПИРАНТСКАЯ ДИССЕРТАЦИЯ\\
}
{\bf \Large	\textsf{ Оптимизация значений параметров в системах компьютерного зрения \\ } }
\end{center}
\vspace{3em}
\begin{multicols}{2}
\vbox to 14 cm{\ }
\noindent 
{\bf Допущена к защите} : {\hrulefill}\\
Заведующий кафедрой МО ЭВМ \\
д.ф.--м.н., проф. \\

\vspace*{1em}
{ \noindent
\hspace*{1.5cm}\hbox to 0cm{\raisebox{-1em}{\small Подпись}}\hspace{-1.5cm}{\hrulefill\ \ Стронгин Р. Г. }
}

\vspace*{1em}
{ \noindent
\hbox to 0cm{\raisebox{0.25em}{<<\qquad \qquad \quad >>}}{\hrulefill\ 2011 г. }
}

\noindent
{\bf Выполнил}: аспирант  \\

\vspace*{0.5em}
{ \noindent
\hspace*{1.5cm}\hbox to 0cm{\raisebox{-1em}{\small Подпись}}\hspace{-1.5cm}{\hrulefill\ \ Морозов А. С. }
}

\vspace{1.5em}
\noindent
{\bf Научный руководитель}: \\
д. т. н., профессор кафедры МО ЭВМ \\

{ \noindent
\hspace*{1.5cm}\hbox to 0cm{\raisebox{-1em}{\small Подпись}}\hspace{-1.5cm}{\hrulefill\ \ ФИО }
}
\end{multicols}
\vfill 
\begin{center} 
{\rm Нижний Новгород \\ 2011 г.} 
\end {center} 
\end{titlepage} 

\tableofcontents

\newpage

\section*{Введение}
\addcontentsline{toc}{section}{Введение} 

Введение.

\section{top}

\section*{Заключение}
\addcontentsline{toc}{section}{Заключение} 

Заключение.


\newpage
\begin{thebibliography}{99}
\addcontentsline{toc}{section}{Список литературы}

	\bibitem{rdtscp} Gabriele Paoloni. How to Benchmark Code Execution Times on Intel IA-32 and IA-64 Instruction Set Architectures, September 2010


\end{thebibliography}

\newpage
\addcontentsline{toc}{section}{Список иллюстраций}
\listoffigures

\newpage
\addcontentsline{toc}{section}{Список таблиц}
\listoftables

%\newpage
%\addcontentsline{toc}{section}{Предметный указатель}
%\documentclass[12pt, a4paper, utf8]{article}
\usepackage[utf8]{inputenc}
\usepackage[english, russian]{babel}
\usepackage[T2A]{fontenc}
%\usepackage{xltxtra}
%\usepackage{xecyr}

%\usepackage{fontspec}
%\defaultfontfeatures{Scale=MatchLowercase}
%\setmainfont[Mapping=tex-text]{Liberation Serif}
%\setsansfont[Mapping=tex-text]{Liberation Sans}
%\setmonofont{Droid Sans Mono}

\usepackage{multicol}

% меняем размеры страницы
\usepackage{geometry}
\geometry{left=3.0cm}
\geometry{right=2.0cm}
\geometry{top=2.0cm}
\geometry{bottom=2.0cm}

\makeindex

% будем считать что переполнения не было, если строка вышла за границу не более чем 0.01pt
\hfuzz=0.001pt

% разрешаю увеличить расстояние между словами не больше чем на 14pt, что бы не было переполнения
\emergencystretch=14pt

% изменение межстрочного интервала
\def\heightline{1.3}
\linespread{\heightline} % 1.3 - это полуторный

%%%%%%%%%%%%%%%%%%%%%%%%%%%%%%%%%%%%%%%%%%%%%%%%
%%% добавляем точки в Оглавление

\renewcommand{\thesection}{\arabic{section}.} 
\renewcommand{\thesubsection}{\arabic{section}.\arabic{subsection}.} 
\renewcommand{\thesubsubsection}{\arabic{section}.\arabic{subsection}.\arabic{subsubsection}.} 

\begin{document}

\begin{titlepage}
\begin{center}
{Министерство образования и науки Российской Федерации\\ 
Государственное образовательное учреждение \\ 
высшего профессионального образования \\ 
<<Нижегородский государственный университет им. Н.И. Лобачевского>>\\

\bf{Факультет вычислительной математики и кибернетики\\
Кафедра математического обеспечения ЭВМ\\
}
\vspace{5em}
АСПИРАНТСКАЯ ДИССЕРТАЦИЯ\\
}
{\bf \Large	\textsf{ Оптимизация значений параметров в системах компьютерного зрения \\ } }
\end{center}
\vspace{3em}
\begin{multicols}{2}
\vbox to 14 cm{\ }
\noindent 
{\bf Допущена к защите} : {\hrulefill}\\
Заведующий кафедрой МО ЭВМ \\
д.ф.--м.н., проф. \\

\vspace*{1em}
{ \noindent
\hspace*{1.5cm}\hbox to 0cm{\raisebox{-1em}{\small Подпись}}\hspace{-1.5cm}{\hrulefill\ \ Стронгин Р. Г. }
}

\vspace*{1em}
{ \noindent
\hbox to 0cm{\raisebox{0.25em}{<<\qquad \qquad \quad >>}}{\hrulefill\ 2011 г. }
}

\noindent
{\bf Выполнил}: аспирант  \\

\vspace*{0.5em}
{ \noindent
\hspace*{1.5cm}\hbox to 0cm{\raisebox{-1em}{\small Подпись}}\hspace{-1.5cm}{\hrulefill\ \ Морозов А. С. }
}

\vspace{1.5em}
\noindent
{\bf Научный руководитель}: \\
д. т. н., профессор кафедры МО ЭВМ \\

{ \noindent
\hspace*{1.5cm}\hbox to 0cm{\raisebox{-1em}{\small Подпись}}\hspace{-1.5cm}{\hrulefill\ \ ФИО }
}
\end{multicols}
\vfill 
\begin{center} 
{\rm Нижний Новгород \\ 2011 г.} 
\end {center} 
\end{titlepage} 

\tableofcontents

\newpage

\section*{Введение}
\addcontentsline{toc}{section}{Введение} 

Введение.

\section{top}

\section*{Заключение}
\addcontentsline{toc}{section}{Заключение} 

Заключение.


\newpage
\begin{thebibliography}{99}
\addcontentsline{toc}{section}{Список литературы}

	\bibitem{rdtscp} Gabriele Paoloni. How to Benchmark Code Execution Times on Intel IA-32 and IA-64 Instruction Set Architectures, September 2010


\end{thebibliography}

\newpage
\addcontentsline{toc}{section}{Список иллюстраций}
\listoffigures

\newpage
\addcontentsline{toc}{section}{Список таблиц}
\listoftables

%\newpage
%\addcontentsline{toc}{section}{Предметный указатель}
%\input{main.ind}

\end{document}



\end{document}



\end{document}



\end{document}

